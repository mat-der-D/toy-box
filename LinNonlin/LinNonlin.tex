\documentclass[12pt, a4j]{jsarticle}
\usepackage[%
top    = 30truemm,%
bottom = 30truemm,%
left   = 25truemm,%
right  = 25truemm]{geometry}
\usepackage{amsmath,amssymb,amsthm,bm,braket,ascmac,enumitem}

\usepackage{mathtools}
\mathtoolsset{showonlyrefs}


\newcommand{\R}{\mathbb{R}}
\newcommand{\e}{\varepsilon}

\begin{document}

次の微分方程式を考える:
\begin{equation}
 \dfrac{d u(t)}{d t}
  = A(t) u(t) + f(t, u(t))
  \label{eq:full_equation}
\end{equation}
ここで 
$u \colon [0, T] \to \R^{n}$,
$f \colon [0, T] \times \R \to \R^{n}$,
$A \colon [0, T] \to \mathrm{Mat}_{n \times n}(\R)$
とする. $A u$ が線形項, $f$ が非線形項であり,
線形項が数値積分の上で硬さの原因になっている状況を考える.

\eqref{eq:full_equation} の線形項のみを取り出した方程式
\begin{equation}
 \dfrac{d v(t)}{d t} = A(t) v(t)
  \label{eq:linear}
\end{equation}
を考える. この方程式の解は明示的に書くことができ,
\begin{equation}
 v(t) = \exp{\left( \int_{0}^{t} A(\tau) d \tau \right)} v(0)
  \label{eq:sol_linear}
\end{equation}
となる. 以下, 記号の簡略化のために
\begin{equation}
 R(t, s) = \exp{\left( \int_{s}^{t} A(\tau) d \tau \right)}
\end{equation}
と置くことにする. これは線形の方程式 \eqref{eq:linear} に従って
時刻 $s$ の状態から時刻 $t$ の状態へ時間発展させる
演算子である. これを用いて, 形式的に
\begin{equation}
 \tilde{u}(t, s) = R(s, t) u(t)
  \quad (s < t)
\end{equation}
とおく. $\tilde{u}(t, s)$ は 
(それが数学的に可能かどうかは別として)
$u(t)$ を線形項のみの方程式 \eqref{eq:linear} を用いて
時刻 $s$ まで ``巻き戻した'' ものに対応する.
$s$ を一旦固定し, $t$ について微分すると
\begin{align}
 \dfrac{\partial \tilde{u}(t, s)}{\partial t}
 &= \dfrac{\partial R(s, t)}{\partial t} u(t)
    + R(s, t) \dfrac{d u(t)}{d t}
 \\
 &= - R(s, t) A(t) u(t) 
    + R(s, t) \left\{ A(t) u(t) + f(t, u(t)) \right\}
 \\
 &= R(s, t) f(t, R(t, s) \tilde{u}(t, s))
 \label{eq:nonlin_part}
\end{align}
となる. もしこの方程式を解くことができれば,
$u(t) = R(t, s) \tilde{u}(t, s)$
により時刻 $t$ の $u(t)$ が求まる. \par

以上の計算方法のポイントは, 線形項に関する積分と
非線形項に関する積分を分離できていることにある.
単に \eqref{eq:full_equation} を Runge-Kutta などで
積分する場合は, 線形項と非線形項を一緒に処理するために,
硬さの原因の線形項を対処できず, 時間刻み幅を気合で
小さくするしかない. 一方で新たな方法では,
$R$ を求める際に (線形の場合には知られていることが多い)
数値的な安定性の高い方法を用いることで,
線形項に対する硬さを回避できる. 敵はもはや非線形項のみであり,
時間刻み幅をさほど細かく取らなくても計算ができる, という寸法である.

とはいえ, $\tilde{u}(t, s)$ という,
``巻き戻し'' を用いて定義された,
必ずしも存在性が保証できない
\footnote{例えば Navier-Stokes のように, 拡散項を含む方程式の場合. 拡散項に対応する偏微分方程式は熱方程式で, ``巻き戻し'' は ill-defined である. なお偏微分方程式の場合は $\R^{n}$ を関数空間に置き換えればよい.}量が間に入っており,
少々扱いが難しい. そこで以下では非線形項由来の方程式
\eqref{eq:nonlin_part} をメジャーないくつかの方法で
差分化し, ``巻き戻し'' の時間発展が実は不要であることを見る. \par

以下, 時間刻み幅を $\e$ とおく.
時刻 $t = s$ での $u(t) = u(s) = \tilde{u}(s, s)$
が与えられたとき, \eqref{eq:nonlin_part} を
$t = s \sim s + \e$ で積分して $\tilde{u}(s + \e, s)$
が得られれば, $u(s + \e) = R(s + \e, s) \tilde{u}(s + \epsilon)$
により $t = s + \e$ における $u(t)$ が得られる.
そこで \eqref{eq:nonlin_part} を $t = s \sim s + \e$ で
数値積分する方法を考える. 説明の簡略化のため,
$F(t, x) = R(s, t) f(t, R(t, s) \tilde{u}(t, s))$
とおく (固定した $s$ は省略).
また記号を濫用し, 数値解でも $u$ や $\tilde{u}$ といった
記号を用いることとする.

\begin{enumerate}[label=(\arabic{*})]
 \item Euler 法 \dots $\e^{1}$ 精度
       \begin{align}
	\tilde{u}(s + \e, s)
	&= \tilde{u}(s, s) + \e F(s, \tilde{u}(s, s))
	\\
	&= u(s) + \e f(s, u(s))
       \end{align}
       と表される. よって
       \begin{align}
	u(s + \e) 
	 &= R(s + \e, s) \, \tilde{u}(s + \e, s)
	 \\
	 &= R(s + \e, s) 
	\left\{ u(s) + \e f(s, u(s)) \right\}	
       \end{align}
       となる. ``巻き戻し'' は起きない.
 \item 中点法 (または修正 (modified) Euler 法) \dots $\e^{2}$ 精度
       \begin{align}
	\tilde{u}(s + \e, s)
	&= \tilde{u}(s, s)
	   + \e F\left( 
	           s + \dfrac{\e}{2},
	           \tilde{u}(s, s) 
	             + \dfrac{\e}{2} F(s, \tilde{u}(s, s))
	         \right)
	\\
	&= u(s)
	   + F\left(
	        s + \dfrac{\e}{2},
	        u(s) + \dfrac{\e}{2} F(s, u(s))
	      \right)
       \end{align}
       より
       \begin{align}
	u(s + \e)
	&= R(s + \e, s) \tilde{u}(s + \e, s)
	\\
	&= R(s + \e, s) u(s) \\
	&\hphantom{{}={}}
	+ \e R\left( s + \e, s + \dfrac{\e}{2} \right)
	  f\left(
	     s + \dfrac{\e}{2},
	     R\left( s + \dfrac{\e}{2}, s \right)
	     G(s)
	   \right),
	\\
	G(s)
	&\coloneqq u(s) + \dfrac{\e}{2} f(s, u(s))
       \end{align}
       となる. ``巻き戻し'' は起きない.

 \item Heun 法 (または改良 (improved) Euler 法) \dots $\e^{2}$ 精度
       \begin{align}
	G_{1}(s) 
	&= F(s, \tilde{u}(s, s))
	 = f(s, u(s)), \\
	G_{2}(s) 
	&= F(s + \e, \tilde{u}(s, s) + \e G_{1}(s))
	\\
	&= R(s, s + \e)
	   f(s + \e,
	     R(s + \e, s) \{ u(s) + \e G_{1}(s) \} ), \\
	\tilde{u}(s + \e, s)
	&= \tilde{u}(s, s)
	   + \dfrac{\e}{2}(G_{1}(s) + G_{2}(s))
	\\
	&= u(s) + \dfrac{\e}{2}(G_{1}(s) + G_{2}(s))
       \end{align}
       である.
       \begin{align}
	\tilde{G}_{1}(s)
	&\coloneqq R(s + \e, s) G_{1}(s)
	= R(s + \e, s) f(s, u(s)),
	\\
	\tilde{G}_{2}(s)
	&\coloneqq R(s + \e, s) G_{2}(s)
	= f(s + \e, R(s + \e, s) u(s)+ \e \tilde{G}_{1}(s))
       \end{align}
       とおけば,
       \begin{align}
	u(s + \e)
	&= R(s + \e, s) \tilde{u}(s + \e, s)
	\\
	&= R(s + \e, s) u(s)
	   + \dfrac{\e}{2}
	   (\tilde{G}_{1}(s) + \tilde{G}_{2}(s))
       \end{align}
       となる. ``巻き戻し'' は起きない.


 \item 4次 Runge-Kutta 法 \dots $\e^{4}$ 精度
       \begin{align}
	G_{1}(s)
	&= F(s, \tilde{u}(s, s)),
	\\
	G_{2}(s)
	&= F\left( 
	      s + \dfrac{\e}{2}, 
	      \tilde{u}(s, s) + \dfrac{\e}{2} G_{1}(s)
	    \right),
	\\
	G_{3}(s)
	&= F\left( 
	      s + \dfrac{\e}{2}, 
	      \tilde{u}(s, s) + \dfrac{\e}{2} G_{2}(s)
	    \right),
	\\
	G_{4}(s)
	&= F(s + \e, \tilde{u}(s, s) + \e G_{3}(s)),
	\\
	\tilde{u}(s + \e, s)
	&= \tilde{u}(s, s)
	+ \dfrac{\e}{6}
	  (G_{1}(s) + 2 G_{2}(s) + 2 G_{3}(s) + G_{4}(s))
       \end{align}
       である.
       \begin{align}
	\tilde{G}_{1}(s)
	&\coloneqq R(s + \e, s) G_{1}(s)
	\\
	&= R(s + \e, s) f(s, u(s)),
	\\
	H_{1}(s)
	&\coloneqq R\left(s + \dfrac{\e}{2}, s \right) G_{1}(s)
	\\
	&= R\left( s + \dfrac{\e}{2}, s \right) f(s, u(s))
	\\
	\tilde{G}_{2}(s)
	&\coloneqq R(s + \e, s) G_{2}(s)
	\\
	&= R\left( s + \e, s + \dfrac{\e}{2} \right)
	   f\left(
	      s + \dfrac{\e}{2},
	      R\left( s + \dfrac{\e}{2}, s \right) u(s)
	      + \dfrac{\e}{2} H_{1}(s)
	    \right)
	\\
	H_{2}(s)
	&\coloneqq R\left( s + \dfrac{\e}{2}, s \right) G_{2}(s)
	\\
	&= f\left(
	      s + \dfrac{\e}{2},
	      R\left( s + \dfrac{\e}{2}, s \right) u(s)
	      + \dfrac{\e}{2} H_{1}(s)
	    \right)
	\\
	\tilde{G}_{3}(s)
	&\coloneqq R(s + \e, s) G_{3}(s)
	\\
	&= R\left( s + \e, s + \dfrac{\e}{2} \right)
	   f\left(
	      s + \dfrac{\e}{2},
	      R\left( s + \dfrac{\e}{2}, s \right) u(s)
	      + \dfrac{\e}{2} H_{2}(s)
	    \right)
	\\
	\tilde{G}_{4}(s)
	&\coloneqq R(s + \e, s) G_{4}(s)
	\\
	&= f\left(
	      s + \e,
	      R(s + \e, s) u(s) + \e \tilde{G}_{3}(s)
	    \right)
       \end{align}
       とおけば,
       \begin{align}
	u(s + \e)
	&= R(s + \e, s) \tilde{u}(s + \e, s)
	\\
	&= R(s + \e, s) u(s)
	+ \dfrac{\e}{6}
	\left(
	\tilde{G}_{1}(s) + 2 \tilde{G}_{2}(s)
	+ 2 \tilde{G}_{3}(s) + \tilde{G}_{4}(s)
	\right)
       \end{align}
       となる. ``巻き戻し'' は起きない.
\end{enumerate}

以上, 時間差分化したあとでは ``巻き戻し'' が問題にならない
ことを見た. 最後に, 実際の実装のステップをまとめる.
非線形項よりも線形項の数値積分に時間がかかる
(例えば陰的解法を用いるために連立方程式を解く必要があるなど)
状況を考えると, 線形項の積分の回数を極力少なくする
必要がある. この点に注意してまとめてみる.

\begin{enumerate}[label=(\arabic*)]
 \item Euler 法
  \begin{enumerate}[label={Step\arabic{*}.}]
   \item $U_{1} = u(s) + \e f(s, u(s))$ を計算
   \item $u(s + \e) = R(s + \e, s) U_{1}$ を計算
  \end{enumerate}
 \item 中点法 (または修正 (modified) Euler 法)
  \begin{enumerate}[label={Step\arabic{*}.}]
   \item $U_{1} = u(s) + \dfrac{\e}{2} f(s, u(s))$ を計算
   \item $U_{2} = R\left(s + \dfrac{\e}{2}, s\right) U_{1}$ を計算
   \item $U_{3} = f\left(s + \dfrac{\e}{2}, U_{2}\right)$ を計算
   \item $U_{4} = R\left(s + \e, s + \dfrac{\e}{2}\right) U_{3}$
	 を計算
   \item $U_{5} = R(s + \e, s) u(s)$ を計算
   \item $u(s + \e) = U_{5} + \e U_{4}$ を計算
  \end{enumerate}
  または次の手順でもよい.
  \begin{enumerate}[label={Step\arabic{*}.}]
   \item $U_{1} = R\left(s + \dfrac{\e}{2}, s\right) u(s)$ を計算
   \item $U_{2} = R\left(s + \dfrac{\e}{2}, s\right) f(s, u(s))$
	 を計算
   \item $U_{3} = U_{1} + \dfrac{\e}{2} U_{2}$ を計算
   \item $U_{4} = f\left(s + \dfrac{\e}{2}, U_{3}\right)$ を計算
   \item $U_{5} = U_{1} + \e U_{4}$ を計算
   \item $u(s + \e) = R\left(s + \e, s + \dfrac{\e}{2}\right) U_{5}$
	 を計算
  \end{enumerate}
 \item Heun 法 (または改良 (improved) Euler 法)
  \begin{enumerate}[label={Step\arabic{*}.}]
   \item $U_{1} = u(s) + \e f(s, u(s))$ を計算
   \item $U_{2} = R(s + \e, s) U_{1}$ を計算
   \item $U_{3} = f(s + \e, U_{2})$ を計算
   \item $U_{4} = u(s) + \dfrac{\e}{2} f(s, u(s))$ を計算
   \item $U_{5} = R(s + \e, s) U_{4}$ を計算
   \item $u(s + \e) = U_{5} + \dfrac{\e}{2} U_{3}$ を計算
  \end{enumerate}
 \item 4次 Runge-Kutta 法
  \begin{enumerate}[label=Step{\arabic*}.]
   \item $U_{1} = u(s) + \dfrac{\e}{6} f(s, u(s))$ を計算
   \item $U_{2} = R(s + \e, s) U_{1}$ を計算
   \item $U_{3} = u(s) + \dfrac{\e}{2} f(s, u(s))$ を計算
   \item $U_{4} = R\left(s + \dfrac{\e}{2}, s\right) U_{3}$ を計算
   \item $U_{5} = f\left(s + \dfrac{\e}{2}, U_{4}\right)$ を計算
   \item $U_{6} = R\left(s + \dfrac{\e}{2}, s\right) u(s)$ を計算
   \item $U_{7} = U_{6} + \dfrac{\e}{2} U_{5}$ を計算
   \item $U_{8} = f\left(s + \dfrac{\e}{2}, U_{7}\right)$ を計算
   \item $U_{9} = \dfrac{\e}{3}(U_{5} + U_{8})$ を計算
   \item $U_{10} = R\left(s + \e, s + \dfrac{\e}{2}\right) U_{9}$
	 を計算
   \item $U_{11} = U_{6} + \e U_{8}$ を計算
   \item $U_{12} = R\left(s + \e, s + \dfrac{\e}{2}\right) U_{11}$
	 を計算
   \item $U_{13} = f\left(s + \e, U_{12}\right)$ を計算
   \item $u(s + \e) = U_{2} + U_{10} + \dfrac{\e}{6} U_{13}$
	 を計算
  \end{enumerate}
\end{enumerate}


\end{document}